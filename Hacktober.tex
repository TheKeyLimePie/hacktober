\documentclass{article}

\title{A hacktober document}
\author{Trololol}
\date{Hacktober 31, 2015}

\usepackage{amsmath}
\usepackage{amssymb}
\usepackage{ngerman}

\title{
    Teh Lolz\\
  \vspace{7mm}
    Ergebnisorientierter {\LaTeX}-Scheincode\\
  \vspace{5mm}
  \small {
    	Eine Publikation \"uber effektive Leistungsoptimierungsstrategien mit Zuhilfenahme von Linear Discriminant Analysis (LDA) in Hinblick auf cremigere K\"aseso\ss{}en auf mexikanischem Maismehl-Salzgeb\"ack\\
    }
  \vspace{1cm}
  \large {
        Jetzt auch auf GitHub!\\
      \vspace{1cm}
    }
}

\begin{document}
	\maketitle	

	\thispagestyle{empty}
	\begin{abstract}
		Lorem ipsum
	\end{abstract}
	
	\newpage
	
	\section{This is complicated stuff}
	\begin{equation*}
		\vec{a} =
		\begin{pmatrix}
		1 \\
		2 \\
		3
		\end{pmatrix}\;\;
		\vec{b} =
		\begin{pmatrix}
		1 \\
		1 \\
		0
		\end{pmatrix}\;\;
		\vec{a} + \vec{b} = \begin{pmatrix}
		2 \\
		3 \\
		3
		\end{pmatrix}
	\end{equation*}
	
	\newpage
	
	\section{This is more complicated stuff}
	\begin{equation}
		\begin{pmatrix}
		1 & 1 & 1 \\
		0 & 0 & 0 \\
		-1 & -1 & -1
		\end{pmatrix}
		\begin{pmatrix}
		a \\
		b \\
		c
		\end{pmatrix} =
		\begin{pmatrix}
		a + b + c \\
		0 \\
		-(a + b + c)
		\end{pmatrix} \>
		a, b, c \in \mathbb{R}
	\end{equation}
\end{document}